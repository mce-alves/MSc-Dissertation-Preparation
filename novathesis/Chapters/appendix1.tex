%!TEX root = ../template.tex
%%%%%%%%%%%%%%%%%%%%%%%%%%%%%%%%%%%%%%%%%%%%%%%%%%%%%%%%%%%%%%%%%%%%
%% appendix1.tex
%% NOVA thesis document file
%%
%% Chapter with example of appendix with a short dummy text
%%%%%%%%%%%%%%%%%%%%%%%%%%%%%%%%%%%%%%%%%%%%%%%%%%%%%%%%%%%%%%%%%%%%

\typeout{NT LOADING FILE appendix1.tex}

\chapter{OCaml Module Signatures}
\label{app:signatures}

\lstset{language=caml, keywordstyle=\color{blue}, caption=Network Module, label=lst:m1}
\begin{lstlisting}
module type Network = sig
    (* the type of blocks included in the messages *)
    type block
    (* the type of proof that is associated with a block *)
    type proof
    (* the type of a transaction *)
    type transaction
    (* the type of the messages sent in the network *)
    type message
    (* the communications channels used for message exchange *)
    type communication_channels
  
    (* minimum delay when sending a message *)
    val min_delay : float
    (* maximum delay when sending a message *)
    val max_delay : float
    (* chance to fail when sending a message *)
    val fail_chance : float
    (* uses the min,max delay bounds, and returns a delay in that interval *)
    val compute_delay : float
    (* sends a message to the provided communication channels *)
    (* A message can be a block*proof, transaction, etc *)
    (* An appropriate channel for the message type will be selected via pattern matching *)
    val send_message : message ->(communication_channels list) ->unit
    (* try to receive a new transaction *)
    val receive_transaction : communication_channels ->(transaction option)
    (* try to receive a block from the network *)
    val receive_block : communication_channels ->((block * proof) option)
end
\end{lstlisting}

\lstset{language=caml, keywordstyle=\color{blue}, caption=Blocktree Module, label=lst:m2}
\begin{lstlisting}
    module type Blocktree = sig
    (* the type of the blocks in the block tree *)
    type block
    (* the type of the hash of a block *)
    type hash
    (* the type of the block tree's implementation *)
    type blocktree
  
    (* the genesis block *)
    val genesis : block
    (* the initial blocktree, containing only the genesis block *)
    val init : blocktree
    (* write a block to the tree *)
    val write_block : block ->blocktree ->blocktree
    (* get the head of the best (heaviest) chain in the blocktree *)
    val get_best_chain_head : blocktree ->block
    (* returns the current best chain *)
    val get_best_chain : blocktree ->(block list)
    (* given a blocktree and two blocks (heads of different chains) *)
    (* returns which should be considered the heaviest chain *)
    val fork_choice_rule : blocktree ->block ->block ->block
    (* ensures that adding the block to the tree won't cause cycles *)
    (* true==no cycles *)
    val no_cycles : blocktree ->block ->bool
    (* checks if the blocktree already contains a block with the same hash *)
    val contains_hash : blocktree ->hash ->bool

end
\end{lstlisting}

\lstset{language=caml, keywordstyle=\color{blue}, caption=Proof of Work Module, label=lst:m3}
\begin{lstlisting}
module type ProofOfWork = sig
    (* the type of the block's implementation *)
    type block
    (* the type of the hash *)
    type hash
    (* the type of the nonce *)
    type nonce
    (* the type of the acceptance threshold *)
    type threshold

    (* the initial nonce that should be used *)
    val init_nonce : nonce
    (* receives a nonce and returns the next one to be used *)
    val next_nonce : nonce ->nonce
    (* computes a hash, given a block and a nonce *)
    val compute_hash : block ->nonce ->hash
    (* accepts (true) or refuses (false) a hash by comparing it to a threshold *)
    val acceptance_function : hash ->threshold ->bool

end
\end{lstlisting}

\lstset{language=caml, keywordstyle=\color{blue}, caption=Proof of Stake Module, label=lst:m4}
\begin{lstlisting}
module type ProofOfStake = sig
    (* the type of the blocktree/blockchain *)
    type blocktree
    (* type of an asynchronous round (slot for which the block will be proposed) *)
    type round
    (* the type of the private and public seeds/keys *)
    type key
    (* the type of the hashes *)
    type hash
    (* type of the structure storing the system's users and their stake *)
    type stakes
    (* the type representing a step in the agreement protocol *)
    type step
    (* the type of the proof produced by the algorithm *)
    type proof
  
    (* returns the users and their corresponding initial stakes (weights) *)
    val weights : stakes
    (* given a peer's private key, a round and step in the protocol, generates a hash and proof if the peer belongs to the committee for that round and step *)
    val committee_sortition : key ->round ->step ->((hash * proof) option)
    (* given a peer's private key and a round, generates a hash and proof if the peer was selected to proposer a block in that round *)
    val proposer_sortition : key ->round ->((hash * proof) option)
    (* given a peer's private key, a round and step in the protocol, and a hash*proof credential, returns whether or not the credentials are valid and belong to the public key's owner *)
    val committee_validation: key ->round ->step ->(hash * proof) ->bool
    (* given a peer's private key, a round in the protocol and a hash*proof credential, returns whether or not the credentials are valid and belong to the public key's owner *)
    val proposer_validation: key ->round ->(hash * proof) ->bool
  
end
\end{lstlisting}