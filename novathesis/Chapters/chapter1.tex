%!TEX root = ../template.tex
%%%%%%%%%%%%%%%%%%%%%%%%%%%%%%%%%%%%%%%%%%%%%%%%%%%%%%%%%%%%%%%%%%%
%% chapter1.tex
%% NOVA thesis document file
%%
%% Chapter with introduction
%%%%%%%%%%%%%%%%%%%%%%%%%%%%%%%%%%%%%%%%%%%%%%%%%%%%%%%%%%%%%%%%%%%

\typeout{NT LOADING FILE chapter1.tex}

\chapter{Introduction}
\label{cha:introduction}

Consensus protocols are notoriously difficult to understand and even harder to define correctly. In the context of blockchain systems, mastering their dynamical behaviour and trust in their correctness is crucial, since the protocols control financial transactions. Moreover, understanding them is essential to enable evolution.

\section{Context}

\todo{not sure how to begin this subsection}

\begin{itemize}
  \item brief overview of what blockchain is and why it is relevant
  \item different aspects involved in developing blockchain technologies
  \item current challenges (verification, validation, simulation, DSLs for clear and concise presentation of the protocols)
\end{itemize}



\section{Problem}

Consensus protocols have crucial roles among distributed systems, such as blockchain technologies. However, there is a lack of specification and experimentation environments to develop consensus and blockchain protocols and thus study their behaviour.

Blockchain protocols are composed of several underlying components, or algorithms, and each of which may have several parameters that can influence the performance and execution of the overall protocol. It is therefore important to be able to reason about the choice of these parameters before deploying the protocol to a real environment.

Hence, there is a need to divide blockchain protocols into essential components, to achieve better abstraction between different protocols, and extract their similarities and key differences. These components can then be used as building blocks to structure the presentation and implementation of the protocols, enhancing their readability and code reusability, and thus enabling the development of an extensible simulation environment.



\section{Goal}

The goal of this project is to develop a simulation environment that allows researchers to make informed decisions about the effects of different parameterizations of the underlying components involved in blockchain protocols.

The simulation environment should be:

\begin{itemize}
  \item modular and extensible, providing the ability to make changes to the protocols, and simulate different families of protocols.
  \item parameterizable, to allow researchers to learn the effects of different parameters in the overall behaviour of the protocol.
  \item structured according to an abstraction framework, such as the five component framework presented in \textit{"Survey Of Distributed Consensus Protocols for Blockchain Networks"} \cite{survey_bchain_networks}, enabling modularity, code reusability and readability.
  \item focused on providing a qualitative evaluation of the simulated protocols.
\end{itemize}

The goal is \textbf{not} to simulate the protocol's behaviour when compared to reality as a mean for it's testing efficiency, but rather to provide researchers results that allow them to reason about certain implementation details, such as: How many adversaries does the system support? Is the system fair when handling stakeholders with different stakes?





\todo{evaluation $/$ validation?}

\todo{mention the DSL "companion" project here?}