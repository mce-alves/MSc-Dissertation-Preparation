%!TEX root = ../template.tex
%%%%%%%%%%%%%%%%%%%%%%%%%%%%%%%%%%%%%%%%%%%%%%%%%%%%%%%%%%%%%%%%%%%%
%% abstrac-en.tex
%% NOVA thesis document file
%%
%% Abstract in English
%%%%%%%%%%%%%%%%%%%%%%%%%%%%%%%%%%%%%%%%%%%%%%%%%%%%%%%%%%%%%%%%%%%%

\typeout{NT LOADING FILE abstrac-en.tex}


Consensus protocols enable the coordination between multiple machines, allowing them to maintain a replicated state and are therefore crucial in building reliable, large-scale distributed systems such as blockchain technologies, which have seen a rise in popularity in recent years.

However, these protocols are often presented in high-level descriptions or "pseudo-code", which interferes with our ability to reason about them and their implementations. Moreover, these protocols are not written in stone, they are constantly evolving. For example, Ethereum uses a blockchain protocol based on proof of work, but is progressively migrating towards an implementation based on proof of stake. Another example is Tezos, where stakeholders that participate in the system are capable of proposing and agreeing on changes to the core protocol itself, allowing it to evolve over time. It is therefore necessary to develop tools to assist in this evolution.

This thesis aims to address the lack of specification and experimentation environments to develop consensus and blockchain protocols, and thus help study their behavior.

The contribution will be a simulation environment that can aid researchers in making informed decisions about the effects of different design choices in the development of these protocols. Simulation of distributed, consensus and blockchain protocols is an active field, however a majority of these simulators are focused on performance evaluation, rather than a qualitative evaluation. The proposed simulation environment will follow a functional Domain Specific Language for abstracting blockchain protocols with the goal of enhancing code readability and reusability, and enabling researchers to reason about the implementation of the protocols. The simulator should be modular and extensible, providing the ability to make changes to the protocols and thus simulate different families of protocols. Finally, the proposed simulator should be parameterizable to allow researchers to learn the effects of different parameters in the overall behavior of the protocol.





% Palavras-chave do resumo em Inglês
\begin{keywords}
Consensus protocols, blockchain, blockchain simulation, extensible simulator, blockchain abstractions
\end{keywords} 
