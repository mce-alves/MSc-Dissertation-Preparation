%!TEX root = ../template.tex
%%%%%%%%%%%%%%%%%%%%%%%%%%%%%%%%%%%%%%%%%%%%%%%%%%%%%%%%%%%%%%%%%%%%
%% chapter4.tex
%% NOVA thesis document file
%%
%% Chapter with lots of dummy text
%%%%%%%%%%%%%%%%%%%%%%%%%%%%%%%%%%%%%%%%%%%%%%%%%%%%%%%%%%%%%%%%%%%%

\typeout{NT LOADING FILE chapter4.tex}

\chapter{Work Plan}
\label{cha:work_plan}

The following chapter presents a tentative planning for the work that will be developed to achieve the goal of the thesis.


\section{Overview}

The project is divided into six major development phases. The timeframes to complete each phase are summarized in figure \ref{fig:gantt}.


\begin{figure}[h]
    \centering

    \resizebox{10cm}{!}{
        \begin{ganttchart}[
            hgrid,
            vgrid={*{6}{draw=none}, dotted},
            bar/.append style={fill=black},
            bar incomplete/.append style={fill=white},
            time slot format=isodate,
            time slot format/base century=2000,
            x unit=0.063cm,
            y unit chart=0.6cm,
            y unit title=1cm, % height of title line and gap
            title height=1, % use full height for title, leaving no gap
            bar top shift=1.15,
            bar height=0.8,
            title label font=\bfseries\normalsize,
            time slot format/start date=2021-03-01]{2021-03-01}{2021-09-30}
        \gantttitle{Work Plan}{214} \\
        \gantttitlecalendar{year, month=shortname} \\
        \gantttitlecalendar[title height=1.6, title label node/.append style={rotate=90}]{week} \\
        \ganttbar{Phase 1}{2021-03-12}{2021-04-04} \\
        \ganttbar{Phase 2}{2021-04-12}{2021-05-16}
        \ganttbar{}{2021-07-26}{2021-08-08} \\
        \ganttbar{Phase 3}{2021-05-17}{2021-06-06}
        \ganttbar{}{2021-06-21}{2021-07-04}
        \ganttbar{}{2021-07-19}{2021-07-25} \\
        \ganttbar{Phase 4}{2021-06-07}{2021-06-20}
        \ganttbar{}{2021-07-05}{2021-07-18} \\
        \ganttbar{Phase 5}{2021-08-09}{2021-08-29}
        \ganttbar{}{2021-09-06}{2021-09-19} \\
        \ganttbar{Phase 6}{2021-04-05}{2021-04-11}
        \ganttbar{}{2021-08-30}{2021-09-05}
        \ganttbar{}{2021-09-20}{2021-09-30} \\
        \end{ganttchart}
    }
    
    \caption{Visualization of the allocated timeframes for each phase of development.}
    \label{fig:gantt}
\end{figure}

\section{Phases of Development}

The following is a more detailed explanation of what will be done in each phase of development.

\vspace{0.25cm}

\textbf{Phase 1 - 3 Weeks}

The first phase will consist in performing further research into the state of the art, building on top of the knowledge already acquired in the preparation phase.

An emphasis will be placed on exploring different simulation techniques and best practices, since this will be crucial for phase two.

Three weeks should be an appropriate amount of time, considering that some level of experimentation will be required to not only learn different simulation techniques, but also to better understand which will be the most appropriate for our use case. This phase may also lead to an overhaul of the function signatures that have been defined.

\vspace{0.25cm}

\textbf{Phase 2 - 6 Weeks}

The second phase will consist in developing the simulator's architecture as well as implementing it's core functionalities, building a solid foundation for phase 3.

More precisely, the result of this phase should be a simulation environment where it is possible to simulate different nodes that are able to communicate among themselves in a network, as well as a mechanism for extracting and aggregating relevant data from each node.

\vspace{0.25cm}

\textbf{Phase 3 - 6 Weeks}

Phase three will involve implementing blockchain specific functionalities. 

The data structures and data structure operations will be implemented, as well as essential operations required to simulate the Bitcoin, Algorand and Ethereum protocols - the necessary block proposal, block validation, information propagation, block finalization and incentive mechanism modules for each of the protocols.

\vspace{0.25cm}

\textbf{Phase 4 - 5 Weeks}

The fourth phase will consist in evaluating the implementations made in the previous phases, through simulation.

The Bitcoin, Algorand and Ethereum protocols will be simulated, and an objective analysis of the results will be performed.

The timeframe of five weeks should be appropriate, to account for possible bug fixing, optimizations and adding new functionalities if needed.

\vspace{0.25cm}

\textbf{Phase 5 - 5 Weeks}

Phase five will be validating the developed simulation environment, through simulating some expected use cases.

The same protocols will be simulated, varying their parameterizations and analysing their results.

\vspace{0.25cm}

\textbf{Phase 6 - 5 Weeks}

Finally, phase six will consist in writting the thesis document and preparing the final presentation.


